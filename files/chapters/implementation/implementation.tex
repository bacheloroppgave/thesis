\chapter{Implementation}
Implementation of the design and architecture is not always straightforward. Especially when working on new unfamiliar frameworks. A lot of work has been done in frameworks such as Angular with MVP, other frameworsk would usually use MVP, or slight alteration of it. This was not the case with React.js and Flux which was chose for this application solution.
\section{Dataflow}
\subsection{React.js}
React focuses on fast rendering of selected components, withouth the need for refreshing the page. Of course it doesnt magically works out of the box, one still needs to design the logick and data transfer in such a way to enable this real-time re-rendering of the components. As discuissed in section \ref{sec:flux} \emph{Flux}, in this application flux architecture was used. This architecture is only a way to write your program, so that there is some structure to follow, and do specific things the same way. An exaple would be handling the user input from a specific component. React has a way to communicate between components, and in the beginnig there were attempts to utilize this in orther to send data between different components, like user input automatically updates some field. It is very simple to achieve in react, although the way to go about it, is not by sending data directly thgrough, and between the components, rather have an external logic that could listen to one components change and update another. It becomes clear when the application grows, it is almost impossible to keep track of all the states and wher the data travells. This leads to the next point - architecture Flux.


\begin{lstlisting}[caption=React setup for listening to cache/data store, label=lst:componentlisteners]

var StudentSelector = React.createClass({

  getInitialState: function() {
    StudentSelectorAPI.getAllStudents();
    return getStateFromStores();
  },

  componentDidMount: function() {
    StudentSelectorStore.addChangeListener(this._onChange);
  },

  componentWillunmount: function() {
    StudentSelectorStore.ChangeListener(this._onChange);
  },

\end{lstlisting}

In the codesnippet above Listing \ref{lst:componentlisteners}, are three handles for when any given component \todo{explain what component is} is rendered initially line 4, it gathers the initial data, line 9 when component is rendered, a new listener is added, it is based on standard JS EventListener \cite{eventlistener}, it listenes to certain actions created by other objects, when such action that this particular component is interested occurs, local private function line 10 \_onChange(), is called. componentWillunmount just removes the listener form the component when the component is not being rendered.
\newpage
\begin{lstlisting}[caption=React render method rendering when the state is updated, label=lst:reactrender]
  render: function() {
    var students = this.state.students;
    var self = this;
    return(
      <div>
        <StudentSelectorSearch
        query={self.state.query}
        searchFor={self._searchFor}
        />
          {
            students.map( function(student) {
              return(
                  <StudentSelectorElement
                  key={student.studentNumber}
                  student={student}
                  handleClick={self._onAddToGroup.bind(self, student)}
                  />
              );
            })
          }
          </ListGroup>
      </div>
    );
  },
\end{lstlisting}

Listing \ref{lst:reactrender}, shows how an example HTML DOM element that would be rendered, it is a simple <div> that wraps other React components, and a .map() function on line 11, that loops through the student data, this data is retrieved from the cache store through the getInitialState() funciton on Listing \ref{lst:componentlisteners} line 4, and with the help of the EventEmitter's listeners that are implemented.
\newpage
\begin{lstlisting} [caption=React private component methods that create actions, label=lst:privatemethodscomponent]
  _onChange: function() {
    this.setState(getStateFromStores());
  },

  _onAddToGroup: function(student) {
    StudentSelectorActionCreators.addStudentToGroup(student);
  },

  _searchFor: function(event) {
    this.setState({query: event.target.value})
    StudentSelectorActionCreators.searchForStudent(event.target.value);
  }
});

Listing \ref{lst:privatemethodscomponent}, shows the pirvate methods of the component, on each call to this.setState() lines 2, 10, a new render cycle is initialized, and new data is being displayed. Those private methods are invoked through Listing \ref{lst:componentlisteners} EventListeners.

\end{lstlisting}


\subsection{Flux}

\section{Websockets}

\section{Backend}

