\chapter{Implementation}
Implementation of the design and architecture is not always straightforward. Especially when working on new unfamiliar frameworks. A lot of work has been done in frameworks such as Angular with MVP, other frameworsk would usually use MVP, or slight alterations of it. This was not the case with React.js and Flux which was chose for this application solution.

The implementation of the design and architecture is not always as straightforward as planned. Especially when working with new unfamiliar frameworks. A lot of work has been done in frameworks for the web, such as AngularJS. These frameworks usually use MVC (Model-view-controller), or slight alterations of it. ReactJS uses Flux, a relatively new way of thinking about data-flow in an application. Flux is unidirectional, which means that all data flows in one way and is controlled by modules such as the Dispatcher. (See diagram \worry{Fix links and figures here}



\section{Dataflow}
\subsection{React.js}

React focuses on fast rendering of selected components, without the need to refresh the page. Traditionally data will be updated whenever the user updates the page. Modern websites rely on data flowing in an asynchronous manner, where new data can come in at any time. Using the architecture of Flux combined with React, it is simple to create websites that update in real time. However, this does not work out of the box, and coding must be done both on the backend and frontend to make this work. As discussed in section \ref{sec:flux} \emph{Flux}, the architecture is only a way to write the program. It is up to the writer to implement the code in a way that works. Imagine an input field on a webpage where you can write your name. The website has a field that greets the visitor upon entering the page, writing their name and pressing a submit button. Normally you would wait for an event triggered by the button, to update the DOM-elements of the input-field and the greeting text. In React, however, you have a Store, where all the data is stored. The greeter-text listens for a change in the store. The user inputs their name, which triggers a change-event in the store, which in turn updates the greeter.

\worry{Legg inn figur om Flux}

{% Graphic for TeX using PGF
% Title: /home/thomas/work/src/github.com/bachopp/thesis/files/chapters/implementation/graphics/simpleGreeterComponent.dia
% Creator: Dia v0.97.3
% CreationDate: Wed Apr 20 16:35:06 2016
% For: thomas
% \usepackage{tikz}
% The following commands are not supported in PSTricks at present
% We define them conditionally, so when they are implemented,
% this pgf file will use them.
\ifx\du\undefined
  \newlength{\du}
\fi
\setlength{\du}{15\unitlength}
\begin{tikzpicture}
\pgftransformxscale{1.000000}
\pgftransformyscale{-1.000000}
\definecolor{dialinecolor}{rgb}{0.000000, 0.000000, 0.000000}
\pgfsetstrokecolor{dialinecolor}
\definecolor{dialinecolor}{rgb}{1.000000, 1.000000, 1.000000}
\pgfsetfillcolor{dialinecolor}
\definecolor{dialinecolor}{rgb}{1.000000, 1.000000, 1.000000}
\pgfsetfillcolor{dialinecolor}
\fill (38.600000\du,0.462500\du)--(38.600000\du,2.362500\du)--(43.550000\du,2.362500\du)--(43.550000\du,0.462500\du)--cycle;
\pgfsetlinewidth{0.100000\du}
\pgfsetdash{}{0pt}
\pgfsetdash{}{0pt}
\pgfsetmiterjoin
\definecolor{dialinecolor}{rgb}{0.000000, 0.000000, 0.000000}
\pgfsetstrokecolor{dialinecolor}
\draw (38.600000\du,0.462500\du)--(38.600000\du,2.362500\du)--(43.550000\du,2.362500\du)--(43.550000\du,0.462500\du)--cycle;
% setfont left to latex
\definecolor{dialinecolor}{rgb}{0.000000, 0.000000, 0.000000}
\pgfsetstrokecolor{dialinecolor}
\node at (41.075000\du,1.607500\du){Store};
\definecolor{dialinecolor}{rgb}{1.000000, 1.000000, 1.000000}
\pgfsetfillcolor{dialinecolor}
\fill (32.941841\du,4.658314\du)--(32.941841\du,6.558314\du)--(37.891841\du,6.558314\du)--(37.891841\du,4.658314\du)--cycle;
\pgfsetlinewidth{0.100000\du}
\pgfsetdash{}{0pt}
\pgfsetdash{}{0pt}
\pgfsetmiterjoin
\definecolor{dialinecolor}{rgb}{0.000000, 0.000000, 0.000000}
\pgfsetstrokecolor{dialinecolor}
\draw (32.941841\du,4.658314\du)--(32.941841\du,6.558314\du)--(37.891841\du,6.558314\du)--(37.891841\du,4.658314\du)--cycle;
% setfont left to latex
\definecolor{dialinecolor}{rgb}{0.000000, 0.000000, 0.000000}
\pgfsetstrokecolor{dialinecolor}
\node at (35.416841\du,5.803314\du){Input field};
\definecolor{dialinecolor}{rgb}{1.000000, 1.000000, 1.000000}
\pgfsetfillcolor{dialinecolor}
\fill (43.631428\du,4.639037\du)--(43.631428\du,6.539037\du)--(48.638928\du,6.539037\du)--(48.638928\du,4.639037\du)--cycle;
\pgfsetlinewidth{0.100000\du}
\pgfsetdash{}{0pt}
\pgfsetdash{}{0pt}
\pgfsetmiterjoin
\definecolor{dialinecolor}{rgb}{0.000000, 0.000000, 0.000000}
\pgfsetstrokecolor{dialinecolor}
\draw (43.631428\du,4.639037\du)--(43.631428\du,6.539037\du)--(48.638928\du,6.539037\du)--(48.638928\du,4.639037\du)--cycle;
% setfont left to latex
\definecolor{dialinecolor}{rgb}{0.000000, 0.000000, 0.000000}
\pgfsetstrokecolor{dialinecolor}
\node at (46.135178\du,5.784037\du){Greeter text};
\pgfsetlinewidth{0.100000\du}
\pgfsetdash{}{0pt}
\pgfsetdash{}{0pt}
\pgfsetmiterjoin
\pgfsetbuttcap
{
\definecolor{dialinecolor}{rgb}{0.000000, 0.000000, 0.000000}
\pgfsetfillcolor{dialinecolor}
% was here!!!
\pgfsetarrowsend{stealth}
{\pgfsetcornersarced{\pgfpoint{0.000000\du}{0.000000\du}}\definecolor{dialinecolor}{rgb}{0.000000, 0.000000, 0.000000}
\pgfsetstrokecolor{dialinecolor}
\draw (35.416841\du,4.658314\du)--(35.416841\du,1.412500\du)--(38.600000\du,1.412500\du);
}}
\pgfsetlinewidth{0.100000\du}
\pgfsetdash{}{0pt}
\pgfsetdash{}{0pt}
\pgfsetmiterjoin
\pgfsetbuttcap
{
\definecolor{dialinecolor}{rgb}{0.000000, 0.000000, 0.000000}
\pgfsetfillcolor{dialinecolor}
% was here!!!
\pgfsetarrowsend{stealth}
{\pgfsetcornersarced{\pgfpoint{0.000000\du}{0.000000\du}}\definecolor{dialinecolor}{rgb}{0.000000, 0.000000, 0.000000}
\pgfsetstrokecolor{dialinecolor}
\draw (43.550000\du,1.412500\du)--(46.135178\du,1.412500\du)--(46.135178\du,4.639037\du);
}}
% setfont left to latex
\definecolor{dialinecolor}{rgb}{0.000000, 0.000000, 0.000000}
\pgfsetstrokecolor{dialinecolor}
\node[anchor=west] at (31.526988\du,3.256956\du){setStore();};
% setfont left to latex
\definecolor{dialinecolor}{rgb}{0.000000, 0.000000, 0.000000}
\pgfsetstrokecolor{dialinecolor}
\node[anchor=west] at (46.733146\du,3.217723\du){getStore();};
\end{tikzpicture}


The above figure shows how the components called Input field and Greeter talks together through the Store. The greeters job is simply to listen for changes in the store, while the input will update the store with the users name. This approach may seem slow and unnecessary, especially on a small application such as the greeter, but the need for a shared Store emerges when the applications grows. React is state-based, meaning it uses states as a way of handling versions of data. New data will update the state of components, which in turn will trigger a re-render of the component. 

As of April 2016, ReactJS is in version 15.01. A new version of ReactJS has been announced, but it is not deployment ready as of now. React is written in JSX, a simplified JavaScript language with XML-layout. The JSX-file is transformed to pure JavaScript. In addition, the XML-attributes are transformed into arguments. Simplified syntax removes a lot of typing, and regular HTML DOM-elements are predefined in React. This means that most HTML-elements can be directly typed in the JSX without having to import or define new elements. Elements such as \emph{input}, \emph{div}, \emph{button} and many more are there out of the box. Below is a simple React component called \emph{SimpleComponent}. It has three methods;


\begin{itemize}
\item \emph{getInitialState:} fired before the component is renderd. This sets up
values for later use. We define a variable called \emph{value}. 
\item \emph{onChange:} fired when the corresponding element is changed. In this case the input-field.
\item \emph{render:} renders the component with a return-statement
\end{itemize}








\begin{lstlisting}[caption=Simple React element with onClick-handlers, label=lst:simplecomponent]

var SimpleComponent = React.createClass({
  getInitialState: function() {
    return {
		value: "",
    }
  },
  onChange: function(event) {
  	this.setState({value: event.target.value});
  }s
  render: function() {
	return(
		<input 
		  type="text"
		  value={this.state.value}
		  onChange={this.onChange}
		/>	
	);  
  }  
\end{lstlisting}

In the example above, the value of the input field is connected to a state called value. The value is updated when the input's onChange-event is triggered (when the user types in the input). ... skriv mer her om basic ReactJS.

Legg også inn annet om Flux her.


\begin{lstlisting}[caption=React setup for listening to cache/data store, label=lst:componentlisteners]

var StudentSelector = React.createClass({

  getInitialState: function() {
    StudentSelectorAPI.getAllStudents();
    return getStateFromStores();
  },

  componentDidMount: function() {
    StudentSelectorStore.addChangeListener(this._onChange);
  },

  componentWillunmount: function() {
    StudentSelectorStore.ChangeListener(this._onChange);
  },

\end{lstlisting}

In the codesnippet above Listing \ref{lst:componentlisteners}, are three handles for when any given component \todo{explain what component is} is rendered initially line 4, it gathers the initial data, line 9 when component is rendered, a new listener is added, it is based on standard JS EventListener \cite{eventlistener}, it listenes to certain actions created by other objects, when such action that this particular component is interested occurs, local private function line 10 \_onChange(), is called. componentWillunmount just removes the listener form the component when the component is not being rendered.
\newpage
\begin{lstlisting}[caption=React render method rendering when the state is updated, label=lst:reactrender]
  render: function() {
    var students = this.state.students;
    var self = this;
    return(
      <div>
        <StudentSelectorSearch
        query={self.state.query}
        searchFor={self._searchFor}
        />
          {
            students.map( function(student) {
              return(
                  <StudentSelectorElement
                  key={student.studentNumber}
                  student={student}
                  handleClick={self._onAddToGroup.bind(self, student)}
                  />
              );
            })
          }
          </ListGroup>
      </div>
    );
  },
\end{lstlisting}

Listing \ref{lst:reactrender}, shows how an example HTML DOM element that would be rendered, it is a simple <div> that wraps other React components, and a .map() function on line 11, that loops through the student data, this data is retrieved from the cache store through the getInitialState() funciton on Listing \ref{lst:componentlisteners} line 4, and with the help of the EventEmitter's listeners that are implemented.
\newpage
\begin{lstlisting} [caption=React private component methods that create actions, label=lst:privatemethodscomponent]
  _onChange: function() {
    this.setState(getStateFromStores());
  },

  _onAddToGroup: function(student) {
    StudentSelectorActionCreators.addStudentToGroup(student);
  },

  _searchFor: function(event) {
    this.setState({query: event.target.value})
    StudentSelectorActionCreators.searchForStudent(event.target.value);
  }
});
\end{lstlisting}


Listing \ref{lst:privatemethodscomponent}, shows the pirvate methods of the component, on each call to this.setState() lines 2, 10, a new render cycle is initialized, and new data is being displayed. Those private methods are invoked through Listing \ref{lst:componentlisteners} EventListeners.




\subsection{Flux}

\section{Websockets}

\section{Backend}

