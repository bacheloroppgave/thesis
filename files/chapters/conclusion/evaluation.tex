\chapter{Evaluation}
\section{Conclusion and discussion}

This report explains the process behind creating a better and more responsive user interface and proposes a new user Autograder front-end system developed at UiS. The report shows the development process, as well as the implementation and design choices.

The first months of developing the new front-end was used to map the functionality of the old Autograder system. To improve the system, we needed to find out it's strengths and weaknesses. Looking at the old Autograder, we found that the process of grading students would be problematic if the number of students grew. The process of grouping students was also cumbersome and not optimal. Using user stories and wireframes, we went though several iterations before landing on the final design of the new front-end. When the wireframes for the systems were designed, we figured that the system would have to be fast and responsive to make it work. With that in mind, we discussed technologies and architectures with the Hein Meling, and settled on the library ReactJS and the architecture Flux.\\Flux has not been used much in the industry, since MVC has been the more preferred architecture. Having worked with Flux and ReactJS, it becomes clear that they are very good for creating application specific software, and user interfaces that are responsive and fast. Since the architecture and frameworks are new for us we had to use some time to learn the implementation and design patterns. We were more familiar with the traditional MVC and pure JavaScript client approach from previous projects, and early prototyping and testing was done to figure out how to work in this new framework would fit into the already existing Autograder system.
\\Going through the old Autograder and the functionality of the system, we found that some parts can be redesigned, so that the users tasks and work flows can be optimised. Heine Furubotten did a great job with the original Autograder front-end, however, we think that some parts of the system should be reworked and redesigned.
\\We wanted to find out if ReactJS and Flux could be used together with a server and a storage solution. Using the dummy data back-end that we created, it was clear that in combination with WebSockets, a fast and responsive site would be possible to create.
\\The end results is a front-end system, that follows the strict Flux architecture, as well as ReactJS guidelines. Most of the functionality of the old Autograder can now be implemented in the new front-end solution, and developed to a point where it can work better than the old front-end.

\section{Future work}
Since a lot of time was used in the initial development process, the prototype lacks some of the functionality that we wanted to implement. Consequently, the site is not deployment-ready as of now. Future work will include making the front-end more reliable, and implement user authentication with the GitHub API. The back-end system must also be replaced with the original Autograder server, or a new one must be made, preferably with WebSockets to get the fast and responsive site that we experienced with the prototype.\\A better API will be written and implemented, so that the server and client can communicate better. Having a front-end that "just works" would be the best possible outcome of this project, but within the limited development time it was not possible. Future work will include tinkering and fixing all the bugs that are present in the prototype, so that it can be ready for the Autograder system for future classes at UiS.
