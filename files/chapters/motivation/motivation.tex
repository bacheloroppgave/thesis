\chapter{Motivation}
For maximum learning effect when learning a new language and working with multiple advanced courses at the same time, quick feedback is the key. Student should be able to get feedback in relatively short time, where it should be easy to see mistakes that were made and ways to improve it. Therefore, is should be easy for the teacher to make comments and give feedback to the students both on and off campus.

The original way of submitting and approving code is cumbersome in many ways and requires time. The original author of the Autograder system wanted this time and effort to be minimized, so that the quality of the feedback can be maximized. The advanced programming courses at the University of Stavanger normally contains a number of assignments with short deadlines, where the feedback students get can depend on a number of factors. The Autograder system is designed to remove some of these factors and effectively remove a lot of overhead work for the teaching staff and students. The program has been in use on a number of advanced programming courses over the last year and informal talk with students and teaching staff has generated a lot of ideas that should be implemented in the new system. This has been taken into consideration and we propose a new Autograder frontend with improved systems both for students, teachers and maintainers in the hope that even more overhead will be removed and the process of grading and giving feedback will be even simpler.

