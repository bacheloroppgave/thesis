\documentclass[12pt,a4paper]{report}
% import packages:
\usepackage{blindtext}
\usepackage{graphicx}		% GRAFIKK (BILDER)
\usepackage[utf8]{inputenc}
\usepackage[T1]{fontenc}
\usepackage{dirtree}
\usepackage{import}
\usepackage{afterpage}
\usepackage{graphicx}
\usepackage{titlesec}
\usepackage[margin=1.0in]{geometry}
\usepackage{listings}
\usepackage{enumitem}
\usepackage{wrapfig}
\usepackage{gensymb}
\usepackage{mathtools}
\usepackage[hidelinks]{hyperref}
\usepackage{pdfpages}
\usepackage{setspace}
\usepackage[english]{babel}
\setcounter{secnumdepth}{4}
\setcounter{tocdepth}{4}
\tolerance = 5000 % LaTeX er normalt streng når det gjelder linjebrytingen.
\hbadness = \tolerance % Vi vil være litt mildere, særlig fordi norsk har så
\pretolerance = 2000
\onehalfspacing
\pagenumbering{gobble}
\usepackage{xfrac}
\usepackage{floatrow}
\usepackage{inputenc}
\setlength{\parindent}{0mm}
\setlength{\parskip}{1em}
\usepackage{tablefootnote}
\usepackage{amsmath}
\usepackage{amsfonts} %Trengs ikke
\usepackage{amssymb} %Trengs ikke
\usepackage{siunitx}
\usepackage{epstopdf}
\usepackage{subfigure}
\newcommand\numberthis{\addtocounter{equation}{1}\tag{\theequation}}



\begin{document}


\chapter*{Motivation}

When larning a new programming language, quick feedback is the key. Student should be able to get feedback in relatively short time, where it should be easy to see mistakes that were made and get relevant feedback from the teacher. A student should be able to submit code in an easy fashion and somehow run and debug the code. The results should be shown in a simple way, where the teacher can add comments and view the code that the students submit. Autograder fills many of the needs that teachers have when grading code. The system uses a frontend webpage where all aspects of the course are managed. This includes groups grouping and grading, as well as reviewing the code of the student and giving feedback. The current frontend is created with a web based system and a server side that handles the logic. The current system completes many of the goals of the author, however, there are some problems that needs to be addressed. One of the goals of the original author is to simplify the process of grading the assignments. The number of students in a course can range from a small class of 10 students to a big class of 30-40 students. The system as of now works great with smaller classes, however, when that number increases, some problems appear. We will therefore revisit these parts of the system and improve on them. We will also predict how the system will be used in the future, so that the frontend can still work even though the backend systems might be rewritten. The system should work both for small classes and bigger classes. The frontend should not be the limiting factor when choosing a programming language. In addition, it should also be a fun and productive experience both for student and teachers, rather than limiting. Furthermore, there have been questions about mobile solutions and other feature of the page. Now that the system is maturing it will be a natural progression to think about mobile solutions. In the ever growing mobile-market, an increasing number of hand-held devices have been introduced to the market, and users will expect this to be integrated in the Autograder system. We will therefore have this in mind when creating the new Autograder frontend.

The advanced programming courses usually consists of multiple assignments with short deadlines. The quality of the feedback can vary depending on a number of factors. To help teachers and students, Autograder should have features for commenting and giving feedback to the student. The original Autograder has been used on a number of courses at bachelor and master level. This enables us to talk to users about the current system, and ways we can improve it. The authors of the new Autograder frontend has already used the system in the course "Operating systems", and have thoughts on how the new system can be improved compared to the old one. We propose a new Autograder frontend system as well as some backend, to not only streamline the work flow for student and teachers, but improve responsiveness of the page.

\end{document}


