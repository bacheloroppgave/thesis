\chapter*{Motivation}
Students should be able to get feedback in relatively short time, where it should be easy to see mistakes that were made and get relevant feedback from the teacher. A student should be able to submit code in an easy fashion and somehow run and debug the code. The results should be shown in a simple way, where the teacher can add comments and view the code that the students submit. Autograder meets many of the needs that teachers have when grading code. The system uses a website where all aspects of the course are managed. This includes grouping of students and grading, as well as links to Github (for code review).

The system in use completes many of the goals of the original author, Furubotten, however, there are some problems that needs to be addressed. One problem is to simplify the process of grading the assignments. This process should be equally straightforward for a course with 500 students as a course with 20. The number of students in a course can range from a small class of 10 students to a big class of 600 students. The system as of now works great with smaller classes, however, when that number increases, problems are expected to emerge. We will therefore revisit these parts of the system and improve them. We will also try to foresee the use of the system, and how the user base may change, so that the front-end can still work independently of future changes. The system should work for both small classes and bigger classes. Moreover, it should not be the limiting factor when choosing a programming language. In addition, it should also be a fun and productive experience both for students and teachers, rather than limiting. There have been some thoughts in the matter of mobile solutions and other features of the website. Now that the system is maturing it will be a natural progression to think about aa mobile-friendly website. In the ever growing mobile-market, an increasing number of hand-held devices have been introduced to the market, and users will expect this to be integrated in the Autograder system. We will therefore have this in mind when creating the new Autograder front-end.

The advanced programming courses usually consists of multiple assignments with short deadlines. The quality of the feedback can vary depending on a number of factors. To help teachers and students, Autograder should have features for commenting and giving feedback to the student. The original Autograder has been used in a number of courses at bachelor and master level. This enables us to talk to users about the current system, about ways in which we can improve it. We have some personal experience with the system as students, and thoughts have been made on how the new system can be improved. We propose a new Autograder frontend system as well as some backend solutions, to not only streamline the work flow for student and teachers, but improve responsiveness of the page.
