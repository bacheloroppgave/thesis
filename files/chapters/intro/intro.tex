%\documentclass[11pt,a4paper]{report}
%\usepackage{blindtext}
%\usepackage[utf8]{inputenc}

%\title{University of Stavanger\\
%Department of Electrical Engineering and Computer Science}
%\author{Thomas Darvik and Tomasz Gliniecki}
%\date{\today}

%\begin{document}
\maketitle
\chapter{Introduction}
%\section*{Introduction}

It is common for students learning programming to go through courses in a practical way with assignments and programming exercises. Assignments and exercises usually consists of a number of tasks that the student will have to solve. When the student is finished, the code is submitted to the teacher. The teacher downloads the code, runs it and grades the student. This approach is time consuming, and by the time the teacher has graded all the exercises, the students have already started on their next assignment. This can lead to students not fully understanding the curriculum and in worst case miss out on key parts of the course. When the courses get more advanced, and the code base increases, this problem gets even bigger. A lot of time is wasted on the overhead work, rather than giving student the help and feedback they need. It also requires a lot of work for the teachers, and can often result in students being left behind by the teacher because of pressure to get all the grading done. Even though the system discussed is in use now, it does not mean that it cannot be changed and improved. Such an improvement has been done at the University of Stavanger, where former master student Heine Furubotten created a system call Autograder. Autograder is a system for code management and grading of code. The system collects student submitted code from Github, a version control system based on Git, and grades it based on test written by the teacher. The system has already been in use on some of the courses held at bachelor and master level at UiS. Autograder requires some basic knowledge of Github and Git, however, these technologies will have to be taught at some point later on anyway. The process of submitting code is easy, as well as the process of getting feedback. 

Automatic testing of code and grading is not a new idea. Many similar systems are in use at other universities. The biggest difference is that they rely on third party software for testing, and can in many cases not fit the need of the teachers and students. The Autograder system is developed at UiS with direct student and teacher feedback. This gives an advantage when it comes to implementation of useful functions, and is therefore tailored for the needs of the advanced programming courses at UiS. Both student and teachers have been very pleased with the system, and Heine speculates that the system have improved results, however, more testing needs to be done to fully conclude this.

When Heine first made the system, he had to focus on many different parts at the same time, in a limited timeframe. 


%\end{document}