\documentclass[11pt,a4paper]{report}
\usepackage{blindtext}
\usepackage[utf8]{inputenc}

\title{University of Stavanger\\
Department of Electrical Engineering and Computer Science}
\author{Thomas Darvik and Tomasz Gliniecki}
\date{\today}

\begin{document}
\maketitle
\chapter{Introduction}
%\section*{Introduction}

[THERE ARE SOME PROBLEMS WITH FRONT-END VS FRONT END -- FIX IT]

Compared to previous times where user interface was largely controlled by the software engineers that wrote them, many modern day systems are created by user experience and designers. Whether the user is a costumer or an employee, the path the user of the program has to take to get from point A to point B has become its own field. This requires the author of the program to more aware of all the aspects of the program they are writing and always think of alternative ways to do the tasks at hand. Equally important is the usability of the program. Modern user interfaces on the web should also take into account that some users might have disabilities or other problems that prevent them from easily navigating the site. 
\\
Most programs go through many changes and iterations throughout the design-process and development phase. Therefore, it can be useful to revisit certain parts and rethink some aspects of the program. At the University of Stavanger, former master student Heine Furubotten (insert reference) created the Autograder-system for automatic code submission and a practical feedback system. The system was designed to remove all the overhead that the students must think of when making their code. In addition it was created so that the teaching staff could do a faster and better job with feedback and learning. The idea of having the code automatically graded and corrected is not a new one, however, at the University of Stavanger, no such systems existed and thus the program was created.
\\
The original Autograder system is created with a web-based front-end system and a server-side system that handles all the assignments and all other tasks of the system. Everyone that uses the system uses the front-end system for managing all aspects of the program, except some key features.


The current design completes many of the goals of the system, however, there are some problems and decisions in the user-interface that can be improved. One of the goals of the original author is to improve the connection that the student and teaching staff have. The new front-end will focus on improved user-experience both for the student and the teaching staff. Streamlining the flow on all aspects and thoroughly going over every part of the system currently in use.
\\


\end{document}