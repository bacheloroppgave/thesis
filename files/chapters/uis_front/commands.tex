% This command just renews what the header of the TOC will be. I prefer "Table of Contents" over "Contents"

\renewcommand{\contentsname}{Table of Contents}

% For referencing figures, tables and equations consistently.

\newcommand{\figref}[1]{Figure~\ref{#1}}	% Referencing a figure
\newcommand{\tabref}[1]{Table~\ref{#1}}		% Referencing a table
\newcommand{\eref}[1]{Equation~\ref{#1}}	% Referencing an equation
\newcommand{\secref}[1]{Section~\ref{#1}}	% Referencing a section

% Command for inserting figure with caption. Requires 5 arguments:
%
%	1) Location/Name of picture file
%	2) Width of picture in document as a fraction of textwidth
%	3) Label of figure (for referencing)
%	4) Figure caption
%	5) Float placement
%

\newcommand{\fig}[5]{
	\begin{figure}[#5]										% Figure placement
		\includegraphics[width=#2\textwidth]{figures/#1}	% Figure size and name
		\caption{#4}										% Figure caption	 (short, long)
		\label{#3}											% Reference label
	\end{figure}
}



% Command for inserting figure without caption (used in front page) and has 3 arguments:
%
%	1) Name of picture file
%	2) Width of picture in document as a fraction of textwidth
%	3) Label

\newcommand{\fignc}[3]{
	\begin{figure}[H]										% Figure placement
		\includegraphics[width=#2\textwidth]{figures/#1}	% Figure size and name
		\label{#3}											% Figure label
	\end{figure}
}


% Command for inserting table with caption. Requires 5 arguments:
%
%	1) Number of and type of columns (i.e. ccc for 3 basic centered columns)
%	2) Label of table (for referencing)
%	3) Table caption
%	4) Float placement
%	5) Table data
%

\newcommand{\tab}[5]{
 \vspace{3mm}							% Bit of spacing before the table
	\begin{table}[#4]					% Table placement

		\caption{#3}						% Table caption (short, long)
	\begin{tabular}{#1}					% Table properties, structure
	#5									% Table data
	\end{tabular}		
	\label{#2}							% Reference label
	\end{table}
}

