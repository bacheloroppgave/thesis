\chapter*{Introduction}
It is common for students that learn programming, to go through courses in a practical way with assignments and programming exercises. This process usually consist of a number of tasks that the student solves. When the assignment is completed, the code is submitted to the teacher. The teacher downloads the code, examines it and grades the student. This approach is time consuming, and by the time the teacher has graded all the exercises, the students could have already started on their next assignment. When the courses get more advanced, and the code base increases, this problem gets even bigger. A lot of time is wasted on the overhead work, rather than providing students with the help and feedback they need. It can often result in students being left behind by the teacher because of pressure to get all the grading done. One solution is to automate the process of code review and testing. Such grading systems have already been implemented at other universities, but these systems are usually hard to configure to the needs of the users. They are comprised of many components that need to interact with each other, which in turn makes the configuration and setup difficult. With that in mind, former master student Heine Furubotten, at the University of Stavanger, created Autograder, which is a system for automatic testing, code reviewing and grading. Autograder collectes student submitted code from GitHub \cite{githubhome}, a version control system, and grades it based on test written by the teacher. The system has already been in use in some of the courses held at bachelor and master level at UiS. Autograder requires some basic knowledge of Github and Git, however, these technologies will have to be taught at some point later in the curriculum anyway. \\Autograder works the same way many systems in the industry do. It implements automatic code testing of submitted work, and mimics the real world scenarios that the student will encounter later through the labs. When students graduate and start working in the industry, a similar process of code review and test-driven development will probably be the most common way of programming. It is therefore not an issue to rely on the students having to learn a version control system. As mentioned earlier Autograder has already been used on some of the advanced programming courses at UiS, and has proven a valuable asset. The system worked as expected and made the job of the teachers easier, by showing all the students in a web-based system. Automatic testing was done though code submitted to Github. The code was automatically downloaded and run on computers at UiS. The build log and results of tests were shown in the  browser. This made it easy for both students and teachers to monitor progress, and evaluate the work. Feedback was given along the way, and helped improve the system with direct feedback from the users. According to Furubotten, the system improved how the assignments and grading process was done. It also made it simpler for the teachers to focus on the tasks at hand, rather than having to think about a cumbersome and slow grading process. \\Automatic testing of code and grading is not a new idea. Similar systems are in use at other universities. The biggest difference is that they rely on third party software for testing, and can in many cases not fit the need of the teachers and students. The Autograder system is developed at UiS with direct student and teacher feedback. Further development of the system is made easy through a process of feedback from the users. Autograder is not limited to any particular programming language, however, it has been tested with Java and the Go programming languages. Both students and teachers have been very pleased with the system, and now that it has been used for some time, relevant user-data can be gathered to improve the system even more. \\In today's mobile market, it is important to have websites that adapt based on screen size. Teachers should be able to use the site from their mobile phones, and grade students on the fly. To make the process of grading students easier, the website must adapt to a number of different types of products, such as smart phones, tablets and desktops.
