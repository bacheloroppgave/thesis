\subsection{Requirements collection and analysis}
\todo{List requirements \worry{section ikke klar}}
\todo{Autograder API,}

Wireframes and user stories together are used in a process of analyzing and refining requirements for the application. Requirements collection and analysis was used while developing Autograder. 
\\In our case, we talked to Professor Hein Meling at UiS, which acted as the client for the new Autograder. Pages such as the results view \worry{ref til figur/wireframe} went though many iterations before landing on the final layout. Wireframes are a great way, one can easily visualize the whole user interface before programming, thus minimizing the amount of rewriting the code.
\\As said in the introduction \worry{ref til introduction}, limitations in the grading process was the main concern. Presenting the wireframes to the client will also make it easier to make rapid changes to the design, using a low-fidelity approach, with sketches and mock-ups. When the initial phase is done, more sophisticated drawings or wireframes are being created. These more detailed drawings with description of areas of the interface, describe what certain elements of the user interface does. Serving as a blueprint for further development. The next step in the process is to reflect the wireframes in code.
\\The client is not supposed to serve as an alternative view of GitHub repositories, GitHub is mainly used as a place to easily store source code, and its login authentication API. The client needs to be able to display courses that a student is enrolled into, assignments that the teacher has published to that course, that also involves group assignments.

